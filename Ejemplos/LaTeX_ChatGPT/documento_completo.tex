\documentclass[12pt]{article}
\usepackage[utf8]{inputenc}
\usepackage[T1]{fontenc}
\usepackage[spanish]{babel}
\usepackage{graphicx}
\usepackage{fancyhdr}
\usepackage{titling}
\usepackage{geometry}
\usepackage{setspace}
\usepackage{lastpage}


% Configuración de márgenes y tamaño de papel
\geometry{
	a4paper,
	left=25mm,
	right=25mm,
	top=30mm,
	bottom=30mm
}

% Configuración de encabezado y pie de página
\pagestyle{fancy}
\fancyhf{}
\fancyhead[L]{\leftmark}
\fancyhead[R]{Autor}
\fancyfoot[L]{Título del documento}
\fancyfoot[R]{\thepage\ de \pageref{LastPage}}

% Configuración de título y autor
\title{Título del documento}
\author{Autor}
\date{Fecha}

% Configuración de interlineado
\onehalfspacing

\begin{document}
	
	% Portada
	\begin{titlepage}
		\begin{center}
			\vspace*{1cm}
			
			\includegraphics[width=0.4\textwidth]{logo.png}
			
			\vspace{1.5cm}
			
			\textbf{\Huge \thetitle}
			
			\vspace{0.5cm}
			
			\textbf{\Large Subtítulo}
			
			\vspace{1.5cm}
			
			\textbf{\Large Autor:} \theauthor \\
			\textbf{\Large Ciudad, País} \\
			\textbf{\Large Fecha:} \thedate
			
			\vfill
			
			\vspace{0.8cm}
			
			\includegraphics[width=0.15\textwidth]{logo.png}
			
			\vspace{1cm}
			
			\textbf{\Large Nombre de la Universidad}
			
		\end{center}
	\end{titlepage}
	
	% Índice
	\tableofcontents
	\newpage
	
	% Introducción
	\section{Introducción}
	
	% Metodología
	\section{Metodología}
	
	% Análisis
	\section{Análisis}
	
	% Conclusiones
	\section{Conclusiones}
	
	% Bibliografía
	\begin{thebibliography}{9}
		\bibitem{ejemplo1}
		Apellido, Nombre.
		\textit{Título del libro}.
		Editorial, Año.
		
		\bibitem{ejemplo2}
		Apellido, Nombre.
		\textit{Título del artículo}.
		Nombre de la revista, Año.
	\end{thebibliography}
	
\end{document}

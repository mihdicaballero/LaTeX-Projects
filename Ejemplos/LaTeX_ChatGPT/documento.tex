\documentclass[12pt]{article}
\usepackage[spanish]{babel}
\usepackage[utf8]{inputenc}
\usepackage{fontenc}
\usepackage{graphicx}
\usepackage{subcaption}
\usepackage{csvsimple}
\usepackage{booktabs}

\usepackage[backend=biber, style=apa]{biblatex}

\addbibresource{biblio.bib}

\begin{document}

\begin{figure}[h]
	\centering
	\includegraphics[width=0.45\textwidth]{figuras/tex_lion}
	\caption{Leyenda de la figura.}
	\label{fig:texlion}
\end{figure}

En la figura \ref{fig:texlion} se puede ver un ejemplo de cómo cargar imágenes en LaTeX.

\begin{figure}[h]
	\centering
	\begin{subfigure}[b]{0.45\textwidth}
		\includegraphics[width=\textwidth]{figuras/tex_lion}
		\caption{Subfigura A.\label{fig:subA}}
	\end{subfigure}
	\hfill
	\begin{subfigure}[b]{0.45\textwidth}
		\includegraphics[width=\textwidth]{figuras/tex_lion}
		\caption{Subfigura B.\label{fig:subB}}
	\end{subfigure}
	\caption{Leyenda de la figura completa.\label{fig:completa}}
\end{figure}

En la figura \ref{fig:completa}, se pueden ver dos subfiguras (ver subfiguras \ref{fig:subA} y \ref{fig:subB}) que muestran cómo cargar imágenes en LaTeX.

\newpage

\begin{table}[ht]
	\centering
	\csvautotabular{tabla.csv}
	\caption{Ejemplo de tabla importada desde archivo CSV}
	\label{tab:tabla-importada}
\end{table}

\begin{table}[h]
	\centering
	\csvreader[    tabular=ccc,    table head=\toprule \textbf{Columna 1} & \textbf{Columna 2} & \textbf{Columna 3} \\ \midrule,    table foot=\bottomrule,    late after line=\\  ]{tabla.csv}{}{ \csvcoli & \csvcolii & \csvcoliii }
	\caption{Mi tabla}
	\label{tab:mi_tabla}
\end{table}

\begin{table}
	\centering
	\begin{tabular}{l c r}
		\toprule
		Ciudad & País & Población \\
		\midrule
		Ciudad de México & México & 8.9 millones \\
		São Paulo & Brasil & 12.3 millones \\
		Buenos Aires & Argentina & 2.9 millones \\
		Lima & Perú & 9.7 millones \\
		Bogotá & Colombia & 7.4 millones \\
		Santiago & Chile & 6.7 millones \\
		Rio de Janeiro & Brasil & 6.7 millones \\
		Caracas & Venezuela & 2.9 millones \\
		Montevideo & Uruguay & 1.3 millones \\
		Quito & Ecuador & 1.6 millones \\
		\bottomrule
	\end{tabular}
	\caption{Ciudades y poblaciones en América Latina}
	\label{tab:ciudades_poblaciones}
\end{table}

\newpage

Esto es una referencia \textcite{ArgosyMedicalAnimation} y \parencite{Bockarjova2020}.


\printbibliography

\end{document}
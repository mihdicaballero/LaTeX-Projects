\documentclass[12pt]{article}
\usepackage[spanish, mexico]{babel}

\usepackage{multirow}
\usepackage{booktabs}
\usepackage[table]{xcolor}
\usepackage{colortbl}
\definecolor{gris}{RGB}{242,242,242}
\usepackage{subcaption}

\usepackage[colorlinks]{hyperref}

\begin{document}
\section{Primer sección}
Esto es un texto de relleno sin sentido. Como vimos en la Tabla \ref{tab:datos}.

\begin{tabular}{|l|c|r|p{3cm}|}
	\hline
	hola & 1 & 1 & 1 \\
	\hline
	1 & hola & 2 & 2 \\
	\hline
	2 & 1 & hola & Esto es un texto más largo \\
	\hline
\end{tabular}

\begin{table}[h]
	\centering
	\caption{La leyenda de la tabla.}
	\begin{tabular}{|l|c|r|p{3cm}|}
		\hline
		hola & 1 & 1 & 1 \\
		\hline
		1 & hola & 2 & 2 \\
		\hline
		2 & 1 & hola & Esto es un texto más largo \\
		\hline
	\end{tabular}
	\label{tab:datos}
\end{table}

\begin{table}[h]
	\centering
	\caption{La leyenda de la tabla.}
	\begin{tabular}{|l|c|r|p{3cm}|}
		\hline
		\textbf{Tema} & \multicolumn{2}{|c|}{\textbf{Valor}} & \textbf{Valor} \\
		\hline
		\multirow[c]{2}{*}{hola} & 1 & 1 & 1 \\
		\cline{2-4}
		 & hola & 2 & 2 \\
		\hline
		2 & 1 & hola & Esto es un texto más largo \\
		\hline
	\end{tabular}
	\label{tab:datos2}
\end{table}

\begin{table}[h]
	\centering
	\caption{La leyenda de la tabla.}
	\rowcolors{1}{white}{gris}
	\begin{tabular}{lcrp{3cm}}
		\toprule
		\textbf{Tema} & \multicolumn{2}{c}{\textbf{Valor}} & \textbf{Valor} \\
		\midrule
		\cellcolor{gris}hola & 1 & 1 & 1 \\
		1 & hola & 2 & 2 \\
		2 & 1 & hola & Esto es un texto más largo \\
		\bottomrule
	\end{tabular}
	\label{tab:datos3}
\end{table}

\begin{table}[h]
	\centering
	\caption{La leyenda de la tabla.}
	\begin{tabular}{lcrp{3cm}}
		\toprule
		\textbf{Tema} & \multicolumn{2}{c}{\textbf{Valor}} & \textbf{Valor} \\
		\midrule
		hola & 1 & 1 & 1 \\
		1 & hola & 2 & 2 \\
		2 & 1 & hola & Esto es un texto más largo \\
		\bottomrule
	\end{tabular}
	\label{tab:datos4}
\end{table}

\begin{table}[h]
	\centering
	\caption{La leyenda de la tabla. Tengo la Tabla \ref{tab:datos5a} y la Tabla \ref{tab:datos5b}.}
	\rowcolors{1}{white}{gris}
	\begin{subtable}{0.4\textwidth}
		\centering
		\caption{La leyenda de la tabla.}
		\rowcolors{1}{white}{gris}
		\begin{tabular}{lcrp{2cm}}
			\toprule
			\textbf{Tema} & \multicolumn{2}{c}{\textbf{Valor}} & \textbf{Valor} \\
			\midrule
			\cellcolor{gris}hola & 1 & 1 & 1 \\
			1 & hola & 2 & 2 \\
			2 & 1 & hola & Esto es un texto más largo \\
			\bottomrule
		\end{tabular}
		\label{tab:datos5a}
	\end{subtable}\hfill
	\begin{subtable}{0.4\textwidth}
		\centering
		\caption{La leyenda de la tabla.}
		\rowcolors{1}{white}{gris}
		\begin{tabular}{lcrp{2cm}}
			\toprule
			\textbf{Tema} & \multicolumn{2}{c}{\textbf{Valor}} & \textbf{Valor} \\
			\midrule
			\cellcolor{gris}hola & 1 & 1 & 1 \\
			1 & hola & 2 & 2 \\
			2 & 1 & hola & Esto es un texto más largo \\
			\bottomrule
		\end{tabular}
		\label{tab:datos5b}
	\end{subtable}
	\label{tab:datos5}
\end{table}

% Table generated by Excel2LaTeX from sheet 'Sheet1'
\begin{table}[htbp]
	\centering
	\caption{Add caption}
	\begin{tabular}{lrrc}
		\toprule
		& \multicolumn{2}{c}{\textbf{Texto}} &  \\
		\midrule
		\multicolumn{1}{c}{\multirow{2}[1]{*}{A}} & 1     & 2     & 3 \\
		& 4     & \cellcolor[rgb]{ 1,  .949,  .8}5 & 6 \\
		hola  & 8     & 7     & 9 \\
		\bottomrule
	\end{tabular}%
	\label{tab:addlabel}%
\end{table}%


\end{document}

\documentclass[12pt]{article}
\usepackage[spanish]{babel}
\usepackage[utf8]{inputenc}
\usepackage{listings}
\usepackage{xcolor}

\definecolor{codegreen}{rgb}{0,0.6,0}
\definecolor{codegray}{rgb}{0.5,0.5,0.5}
\definecolor{codepurple}{rgb}{0.58,0,0.82}
\definecolor{backcolour}{rgb}{0.95,0.95,0.92}

\lstdefinestyle{mystyle}{
	backgroundcolor=\color{backcolour},   
	commentstyle=\color{codegreen},
	keywordstyle=\color{magenta},
	numberstyle=\tiny\color{codegray},
	stringstyle=\color{codepurple},
	basicstyle=\ttfamily\footnotesize,
	breakatwhitespace=false,         
	breaklines=true,                 
	captionpos=b,                    
	keepspaces=true,                 
	numbers=left,                    
	numbersep=5pt,                  
	showspaces=false,                
	showstringspaces=false,
	showtabs=false,                  
	tabsize=2
}

\lstset{style=mystyle}

\begin{document}
	
Código a mano simple con \verb|hola|.

También como un bloque de código con texto monoespaciado.
\begin{verbatim}
	import math
	import numpy as np
	from lib.analytical import csa
\end{verbatim}	

\lstlistoflistings
	

Código en Python

\begin{lstlisting}[language=Python, caption={Código en Python.}]
"""
---------
sin2_theta  = np.sin(theta)**2
"""
import math
import numpy as np
from lib.analytical import csa

sin2_theta  = np.sin(theta)**2
+= -= *= /= + - * / ? < > & % == <=
# += -= *= /= + - * / ? < > & % == <=
def test(a=100, b=True):
<= >= == 2 + 3j * 7e-3
\end{lstlisting}

Código en Octave

\begin{lstlisting}[language=Octave]
% Funcion que calcula los grados de libertad de un elemento 
% en funcion de las conectividades

function gdl = conec2gdlframe (conectiv , elem )

nnod = size(conectiv,2); 
gdl = zeros(1,3*nnod);
for i=1:nnod
	gdl(3*i-2) = 3*conectiv(elem,i)-2;
	gdl(3*i-1) = 3*conectiv(elem,i)-1;
	gdl(3*i) = 3*conectiv(elem,i);
end
\end{lstlisting}

Cargo archivo de octave de la carpeta.
\lstinputlisting[language=Octave]{codigo/moment.m}


\end{document}

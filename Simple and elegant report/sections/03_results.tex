\section{Resultados}

Sobre la mesa había desparramado un muestrario de paños - Samsa era viajante de comercio-, y de la pared colgaba una estampa recientemente recortada de una revista ilustrada y puesta en un marco dorado. La estampa mostraba a una mujer tocada con un gorro de pieles, envuelta en una estola también de pieles, y que, muy erguida, esgrimía un amplio manguito, asimismo de piel, que ocultaba todo su antebrazo.

Gregorio miró hacia la ventana; estaba nublado, y sobre el cinc del alféizar repiqueteaban las gotas de lluvia, lo que le hizo sentir una gran melancolía. «Bueno -pensó-; ¿y si siguiese durmiendo un rato y me olvidase de todas estas locuras? » Pero no era posible, pues Gregorio tenía la costumbre de dormir sobre el lado derecho, y su actual estado no le permitía adoptar tal postura.

Por más que se esforzara volvía a quedar de espaldas. Intentó en vano esta operación numerosas veces; cerró los ojos para no tener que ver aquella confusa agitación de patas, que no cesó hasta que notó en el costado un dolor leve y punzante, un dolor jamás sentido hasta entonces. - ¡Qué cansada es la profesión que he elegido! -se dijo-.

Siempre de viaje. Las preocupaciones son mucho mayores cuando se trabaja fuera, por no hablar de las molestias propias de los viajes: estar pendiente de los enlaces de los trenes; la comida mala, irregular; relaciones que cambian constantemente, que nunca llegan a ser verdaderamente cordiales, y en las que no tienen cabida los sentimientos. ¡Al diablo con todo! Sintió en el vientre una ligera picazón. Lentamente, se estiró sobre la espalda en dirección a la cabecera de la cama, para poder alzar mejor la cabeza. Vio que el sitio que le picaba estaba cubierto de extraños untitos blancos. Intentó rascarse con una pata; pero tuvo
\section{Base de datos de bibliografía}

Una base de datos bibliográfica es un archivo que contiene información sobre artículos, libros y otros documentos. El objetivo principal de esta base de datos es que debe contener datos sobre estos documentos pero nada sobre la presentación visual.

Cada usuario debe crear su propia base de datos Bib\LaTeX\ de cada documento al que desea hacer referencia o puede querer hacer referencia en el futuro. La mejor manera de administrar la base de datos es usando un software que se dedique a la gestión de referencias bibliográficas. Ejemplos de éstos son: \hyperref{https://www.mendeley.com/?interaction_required=true}{}{}{Mendeley}, \hyperref{https://www.zotero.org/}{}{}{Zotero}, \hyperref{https://www.bibme.org/}{}{}{BibMe}. Hay varios gestores más, gratuitos y pagos, que el usuario puede investigar por su cuenta.

\subsection{Estructura de la base de datos}
Una base de datos bibliográfica es una colección de registros, cada uno de los cuales proporciona información sobre una publicación. Cada registro se ve así:
\begin{lstlisting}
	@kind {key,
		info = value,
		info = value,
		...
	}
\end{lstlisting}
\noindent
en donde:
\begin{description}
	\item[@kind] especifica qué tipo de documento es, es decir, si es un libro, un artículo, una tesis o lo que sea. 
	\item[key] es la clave o etiqueta del documento que se utiliza para referirse al documento. Esto lo define cada uno.
	\item[info] indica un aspecto de la información con respecto a este documento, como autor, título, editor u otra cosa.
	\item[value] es la información real proporcionada. Normalmente se cita en llaves o signos de comillas dobles:
\end{description}
\begin{lstlisting}
	title = {Nombre de documento}
	title = "Nombre de documento"
\end{lstlisting}
Números (como un año) o abreviaturas no deberían ir con comillas. 

\subsection{Tipos de documentos}

A continuación en la \autoref{tab:tipos} se presenta una lista de los tipos más comunes de documentos que podemos llegar a tener:
\begin{table}[htbp]
	\centering
	\caption{Tipos de documentos}
	\begin{tabular}{ll}
		\toprule
		\textbf{@article} 		& Un artículo de revista. \\
		\textbf{@book} 			& Un libro publicado. \\
		\textbf{@booklet} 		& Como un libro, pero sin editor. \\
		\textbf{@inproceedings} & Artículo en actas de congresos. \\
		\textbf{@manual} 		& Documentación técnica. \\
		\textbf{@misc} 			& No se ajusta a ningún otro tipo. \\
		\textbf{@online} 		& Una página web u otro recurso en línea. \\
		\textbf{@reference} 	& Un diccionario o similar. \\
		\textbf{@report} 		& Informe de investigación o similar. \\
		\textbf{@thesis} 		& Cualquier tipo de tesis. \\
		\textbf{@unpublished} 	& Todavía no publicado. \\
		\bottomrule
	\end{tabular}%
	\label{tab:tipos}%
\end{table}%

\subsection{Nombres de autor}

El nombre del autor o autores es quizás el elemento de información más importante en una bibliografía. Normalmente, simplemente enumera sus nombres con "\textbf{and}"\ entre cada uno. (La palabra "\textbf{and}"\ se sustituirá automáticamente por una coma o la palabra adecuada en el idioma del documento). La forma de definirlo es la siguiente:

\begin{lstlisting}
	author = {name1 and name2 and name3}
\end{lstlisting}

Si la publicación tiene autores adicionales que no se conocen, se puede terminar la lista con la palabra "\textbf{others}":

\begin{lstlisting}
	author = {name1 and name2 and others}
\end{lstlisting}

Cada nombre se indica de la siguientes dos formas:
\begin{itemize}
	\item \textbf{Nombre Apellido}
	\item \textbf{Apellido, Nombre}
\end{itemize}

También se puede registrar solo el \textbf{Apellido} y el nombre puede ser la primer letra, por ejemplo:

\begin{lstlisting}
	author = {Costabel, M and Verde, J}
\end{lstlisting}

El gestor de referencias bibliográficas ya hace esto correctamente, lo que importa es tener bien registrados los datos de los autores en dicho software, y en caso de querer ajustar a mano, se puede editar el archivo \verb|.bib| de la bibliografía de nuestro documento.

\subsection{Información adicional}
En la \autoref{tab:tipos_info} se describe las opciones de informaciones más comunes que se tienen en un tipo de documento. Se debe proveer de toda la información posible en cada referencia y luego Bib\LaTeX\ usará la información que sea relevante.

% Table generated by Excel2LaTeX from sheet 'Sheet1'
\begin{table}[h]
	\centering
	\caption{Tipos de información en base de datos.}
	\begin{tabular}{ll}
		\toprule
		\textbf{author } & El nombre del autor. \\
		\textbf{chapter} & El capítulo particular. \\
		\textbf{date } & La fecha de publicación (como aaaa-mm-dd). \\
		\textbf{edition } & La edición (como un número). \\
		\textbf{institution } & Institución empresarial o académica. \\
		\textbf{isbn } & International Standard Book Number. \\
		\textbf{issn } & Número de serie estándar internacional \\
		\textbf{journaltitle } & El titulo de la revista. \\
		\textbf{keywords } & Una lista de palabras clave separadas por comas para la entrada. \\
		\textbf{location } & Donde reside el editor o la institución. \\
		\textbf{month } & El mes de publicación (como un número o una abreviatura). \\
		\textbf{note } & Datos adicionales. \\
		\textbf{pages } & Qué paginas se hacen referencia. \\
		\textbf{publisher } & La editorial \\
		\textbf{subtitle } & El subtítulo del documento. \\
		\textbf{title } & El título del documento. \\
		\textbf{type } & El tipo específico (por ejemplo, de una @thesis). \\
		\textbf{url } & Una dirección web. \\
		\textbf{urldate } & Cuando se accedió al documento (como aaaa-mm-dd). \\
		\textbf{version } & Un número de versión (como un número). \\
		\textbf{year } & El año de publicación. \\
		\bottomrule
	\end{tabular}%
	\label{tab:tipos_info}%
\end{table}%

Los nombres de los meses vienen predefinidos de la siguiente manera:
%
\begin{lstlisting}
	jan, feb, mar, apr, may, jun, jul, aug, sep, oct, nov, dec
\end{lstlisting}

Éstos comandos se convierten en la palabra completa según el idioma de referencia. Es importante que estén escritos \textbf{sin} corchetes, solo la abreviatura.


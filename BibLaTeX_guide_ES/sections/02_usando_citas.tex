\section{Usando citas}

\subsection{El paquete Bib\LaTeX}

Para poder utilizar Bib\LaTeX\ \footnote{Para ver toda la documentación de este paquete, ir al siguiente \hyperref{https://www.ctan.org/pkg/biblatex}{}{}{link}.} se debe importar el paquete, con el siguiente comando:
\begin{lstlisting}
	\usepackage[options]{biblatex}
\end{lstlisting}

Las opciones más comunes para el paquete se listan en la \autoref{tab:opciones}.

% Table generated by Excel2LaTeX from sheet 'Sheet1'
\begin{table}[htbp]
	\centering
	\caption{Opciones más comunes de Bib\LaTeX.}
	\begin{tabular}{ll}
		\toprule
		\textbf{backend=biber} & Usa biber como backend (es el recomendado). \\
		\textbf{bibencoding=utf8} & Se especifica la codificación del bib. \\
		\textbf{maxbibnames=n} & Máximo número de autores listados en la bibliografía, default=3. \\
		\textbf{minbibnames=n} & Mínimo número de autores listados en la bibliografía, default =1. \\
		\textbf{sortcites} & Ordenar citas numéricas. \\
		\textbf{style=xxx} & Se especifica que estilo de bibliografía usar. \\
		\textbf{url=true/false} & Indica si se quiere ver o no el url en la referencia. \\
		\bottomrule
	\end{tabular}%
	\label{tab:opciones}%
\end{table}%


\subsection{El comando \textbackslash addbibresource}

Este comando lista todos los archivos de bilbiografías que se quieran cargan. Estos suelen ser generados automáticamente por un gestor de biblbiografías.

\begin{lstlisting}
	\addbibresource{proyecto.bib}
	\addbibresource{tesis.bib}
\end{lstlisting}

Notar que es importante que los archivos estén escritos con su extensión \verb|.bib| para poder ser cargados correctamente.

\subsection{El comando \textbackslash cite}

Este comando se utilza cuando se quiere referencias a un documento:
\begin{lstlisting}
	... creado por Lesie Lamport. \cite{LaTeX}
\end{lstlisting}

También se pueden referencias varios documentos en una misma cita:
\begin{lstlisting}
	... de los autores. \cite{autor1, autor2, autor3}
\end{lstlisting}

Es mejor poner el comando \verb|\cite{}| luego de un punto final, y no antes. 

\subsubsection{Información adicional en una cita}
Se puede proveer de información adicional al citar un documento si se especifica un sufijo o prefijo como opciones del comando \verb|\cite{}|. 
\begin{lstlisting}
	\cite[prefix info][postfix info]{key}
\end{lstlisting}

El prefijo o sugijo que se escriba como un número, tanto en números romános o arábicos, será tratado como una página del documento.

\begin{table}[h]
	\centering
	\caption{Opciones del comando \textbackslash cite{}}
	\begin{tabular}{ll}
		\toprule Comando				& Cita \\ 
		\midrule \verb|\cite{guia1}| 	& 	\cite{guia1} \\
		\verb|\cite[10]{guia1}| 	& 	\cite[10]{guia1} \\
		\verb|\cite[Ver también][xiv]{guia1}| 	& 	\cite[Ver también][xiv]{guia1} \\
		\bottomrule
	\end{tabular}
\end{table}

\clearpage
\subsubsection{Variantes del comando \textbackslash cite}

Se pueden usar variaciones de forma tal de proveer de información diferente o con otro formato sobre un documento. Las distintas opciones se muestran en la \autoref{tab:cites}. 

\begin{savenotes}
	\begin{table}[h] 
		\centering
		\caption{Opciones del comando \textbackslash cite{}}
		\begin{tabular}{ll}
			\toprule Comando				& Cita \\ 
			\midrule \verb|\citeauthor{guia1}| 	& 	\citeauthor{guia1} \\
			\verb|\citetitle{guia1}| 	& 	\citetitle{guia1} \\
			\verb|\citeurl{biber}| 		& 	\citeurl{biber} \\
			\verb|\citeyear{guia1}| 	& 	\citeyear{guia1} \\
			\verb|\footcite{guia1}| 	& 	Como una nota al pie.\footcite{guia1} \\
			\verb|\parencite{guia1}| 	& 	\parencite{guia1} \\
			\verb|\textcite{guia1}| 	& 	\textcite{guia1} \\
			\bottomrule
		\end{tabular}
		\label{tab:cites}
	\end{table}
\end{savenotes}

Es recomendable usar estos comandos para citar al autor o una publicación, así se mantiene la consistencia de la apariencia del documento. 

\subsubsection{El comando \textbackslash nocite}

Con el comando \verb|\nocite{}| se puede incluir una entrada en la lista de referencias bibliográficas sin tener que citarla en el cuerpo del texto. Con el comando \verb|\nocite{*}| se incluyen todas las entradas del documento \verb|.bib| cargado en la lista de referencias.

\subsection{El comando \textbackslash printbibliography}
Este comando se coloca en donde se quiera que aparezca la lista de referencias bibliográficas. Usualmente se encuentra al final del documento, pero podría ir, por ejemplo, al final de cada capítulo de un libro.

\begin{lstlisting}
	\printbibliography[options]
\end{lstlisting}

Se pueden indicar opciones para controlar la apariencia de las referencias bibliográficas. En la \autoref{tab:printbib} se indican las más comunes.

\begin{table}[h] 
	\centering
	\caption{Opciones del comando \textbackslash cite{}}
	\begin{tabular}{lp{11cm}}
		\toprule Opción					& 	Resultado \\ 
		\midrule \verb|heading=name| 	& 	Cambia el nombre por defecto. La opción "bibintoc"\ hace que aparezca en tabla de contenidos y "bibnumbered"\ que aparezca numerado. \\
		\verb|keyword=keyword| 			& 	Solo imprimir entradas con el keyword indicado \\
		\verb|notkeyword=keyword| 		& 	lo opuesto a la opción de keyword \\
		\verb|prenote=name|		 		& 	imprime una nota al final de las Referencias\\
		\verb|title=text|		 		& 	Reemplaza el título por defecto de "Referencias" por uno indicado. \\
		\verb|type=entry type| 			& 	Limita la lista de entradas a la de los tipos indicados (book, article, phdthesis, etc.)\\
		\verb|nottype=entry type| 		& 	Lo opuesto a la opción anterior \\
		\bottomrule
	\end{tabular}
	\label{tab:printbib}
\end{table}

\subsection{Cantidad de autores citados}
Si la lista de autores de un documento es muy larga, Bib\LaTeX solo imprime los primeros y luego coloca "et al"\. La cantidad de autores que definen cuando es "muy largo"\ se puede definir dentro de las opciones del paquete Bib\LaTeX. Se modifica con las siguientes opciones:
\begin{description}
	\item[maxbibnames=n] especifica la cantidad máxima de autores como \textbf{n}. Esto quiere decir que si hay más de \textbf{n} autores, la lista se abrevia. El valor por defecto de \textbf{n} es 3.
	\item[minbibitems=n] controla cuantos autores son nombrados en una lista abreviada. El valor por defecto es 1.
\end{description}

Por ejemplo, si se quiere que la lista se acorte cuando la cantidad de autores supere los 5 y que en tal caso se indiquen los primeros tres autores, hay que escribir lo siguiente:
\begin{lstlisting}
	\usepackage[maxbibnames=5,minbibnames=3,...]{biblatex}
\end{lstlisting}

